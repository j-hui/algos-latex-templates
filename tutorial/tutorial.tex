%%%% BEGIN document formatting %%%%
\documentclass[11pt]{article}
\usepackage{fullpage}
\usepackage{setspace}
\onehalfspacing
\setlength{\topmargin}{-0.7in}
\setlength{\textwidth}{6.5in}
\setlength{\oddsidemargin}{0.0in}
\setlength{\textheight}{10.0in}
\setlength{\parindent}{0in}
\pagestyle{empty}
%%%% END document formatting %%%%

%%%% BEGIN math packages %%%%
\usepackage{amsmath}
\usepackage{amssymb}
\usepackage{mathtools}
%%%% END math packages %%%%

%%%% BEGIN pseudocode packages %%%%
\usepackage{algpseudocode,algorithm}
\usepackage{listings}
\lstset{
  columns=fixed,
  basicstyle=\ttfamily, % fixed-width font
}
%%%% END pseudocode packages %%%%

%%%% BEGIN other packages %%%%
\usepackage{hyperref}
%%%% END other packages %%%%

% This defines a new command which you can use to start a new problem item
% when under the enumerate environment, e.g.:
%
%   \problemitem{2}
%
%     Answer to problem 2...
%
\newcommand{\problemitem}[1]{
  \bigskip
  \item {\bf Solution to problem #1}
  \medskip
}

\newcommand{\BACKSLASH}{\char`\\ }


\begin{document}

\begin{flushright}
{\bf CSOR W4231.001---Summer 2019}
\end{flushright}
\begin{flushleft}
  Name: Example Jones\\
  UNI: eg2019\\
  Session: {\bf X} (example) \\ % keep the relevant choice
  Collaborators: \\ % list names and UNIs of all collaborators
\end{flushleft}

\bigskip
\centerline{\bf \LaTeX\ Tutorial}

\begin{enumerate}

\problemitem{1}

{\large \textbf{Writing text}}

To write plain text, simply write it as is.

If you leave at least blank line, you will start a new paragraph.
But if you start immediately on the next line, the text will be included in the
same paragraph.

To \textbf{write in bold}, use the \texttt{\BACKSLASH textbf} command. To
\textit{write in} \emph{italics}, use the \texttt{\BACKSLASH textit} command or the
\texttt{\BACKSLASH emph} command. To write in \texttt{teletype (fixed width)
text}, use the \texttt{\BACKSLASH texttt} command.

\problemitem{2}

{\large \textbf{Lists}}

To write bulleted lists, write each \texttt{\BACKSLASH item} under the
\texttt{itemize} environment:

\begin{itemize}
  \item Like
  \item This.

    You can still write separate paragraphs under the same bullet.
  \item

    Though as usual, items aren't
particularly sensitive
    to whitespace.


\end{itemize}

To write numbered lists, write each \texttt{\BACKSLASH item} under the
\texttt{enumerate} environment:
\begin{enumerate}
  \item Like
  \item So.
  \item These items are indexed with (a), (b), etc. because the problem numbers
    are indexed with the arabic numerals, 1., 2., etc.
\end{enumerate}

To write lists with descriptions, write each \texttt{\BACKSLASH item} under the
\texttt{description} environment, with the description in [square brackets]:
\begin{description}
  \item[Foo] Like
  \item[Bar] This.
  \item[Baz]
    You can also write descriptions in square brackets in an
    \texttt{enumerate} list to replace the numeric/alphabetical indices, but the
    description won't be in bold.
\end{description}

\problemitem{3}

{\large \textbf{Writing math}}

To write math text inline, use the dollar sign (\texttt{\$}).  When in math
mode, writing appears $like this$, with spaces removed.  You will need to
manually $escape\ spaces$ using a backslash followed by a space. Inline math is
great for referencing variables like $x$ and $\sigma$ and $L_i$ in prose
writing.

To write displayed equations on their own line, write them enclosed by
\texttt{\BACKSLASH[} and \texttt{\BACKSLASH]}, e.g.:
\[
  E = mc^2
\]

Here are some handy math symbols:
\begin{description}
  \item[Superscript]
    \[
      a^b x^{2n}
    \]
  \item[Subscript]
    \[
      a_b x_{ij}
    \]
  \item[Logarithms]
    \[
      \log n
    \]
  \item[Infinity]
    \[
      \infty
    \]
  \item[Relational operators]
    \[
      = \neq < \leq > \geq
    \]
  \item[Set operators]
    \[
      \in \notin \{ \} \emptyset \subset \subseteq \cup \cap
    \]
  \item[Logical operators]
    \[
      \wedge \vee \rightarrow \neg
    \]
  \item[Logical metalanguage]
    \[
      \Rightarrow \Leftrightarrow \equiv \doteq
    \]
  \item[Sets]
    \[
      \mathbb{N} \subset \mathbb{Z} \subset \mathbb{R}
    \]
  \item[Greek]
    \[
      \delta \alpha \beta \omega \Omega \theta \Theta \pi \Gamma
    \]
  \item[Fractions (vertical)]
    \[
      \frac{a}{b}
    \]
  \item[Min and max]
    \[
      \min_{v \in V} x_v + \max_{u \in U} x_u
    \]
  \item[Arbitrary text]
    \[
      x + \text{lorem ipsum}
    \]
  \item[Sums]
    \[
      \sum_{i = 0}^{n} \sum_{j = 0}^{n} x_{ij}
    \]
  \item[Big parens]
    \[
      s = \left( \sum_{i = 0}^{\infty} x/i \right)
    \]
  \item[Cases]
    (Use \texttt{\&} to align columns, and use \texttt{\BACKSLASH \BACKSLASH} to
    start new lines.)
    \[
      f(x) = \begin{cases}
        1                     & \text{if } x \leq 1\\
        f(x - 1) + f(x - 2)   & \text{otherwise}
      \end{cases}
    \]
  \item[Aligned, multi-line equations]
    (Use \texttt{\&} to align columns, and use \texttt{\BACKSLASH \BACKSLASH} to
    start new lines.)
    \[
      \begin{aligned}
        f = & x_i + y_i + 2z_i \\
            & 5u + 8
      \end{aligned}
    \]
\end{description}

\newpage

\problemitem{4}

{\large \textbf{Writing pseudocode}}

There are two ways to write pseudocode: using \texttt{algorithmic}, or using
\texttt{lstlisting}.

\medskip
\textbf{Using \texttt{algorithmic}}

Psuedocode written using the \texttt{algorithmic} environment will appear more
like the code you see in your lecture slides and textbook, e.g.:

\begin{algorithmic}
  \Function{Fib}{$n$}
    \If{$n \leq 1$}
      \State \textbf{return} $1$
    \Else
      \State \textbf{return} \textsc{Fib}($n - 1$) $+$ \textsc{Fib}($n - 2$)
    \EndIf
  \EndFunction
\end{algorithmic}

For more examples and details, see:

\url{http://tug.ctan.org/macros/latex/contrib/algorithmicx/algorithmicx.pdf}.

\medskip
\textbf{Using \texttt{lstlisting}}

If you prefer to write your pseudocode in plaintext, you may use the
\texttt{lstlisting} environment, e.g.:

\begin{lstlisting}
  def fib(n):
    if n <= 1:
      return 1
    else:
      return fib(n - 1) + fib(n - 2)
\end{lstlisting}


\problemitem{5}

{\large \textbf{Tables}}

You shouldn't need to use this too often, but here's a simple template to get
you started. The alignment rules are the same as \texttt{cases} and
\texttt{aligned}: inside the \texttt{tabular} environment, \texttt{\&} is used
to align columns, and \texttt{\BACKSLASH \BACKSLASH} is used to start new rows.

\begin{table}[tbh]
  \centering
  \renewcommand\arraystretch{1.5}
  \begin{tabular}{|c|c|c|c|c|}\hline
    1 & 2 & 3 & 4 & 5\\ \hline % nice and aligned
  1   & 2 &  3 3 3 3 3 3 3 3 3 3     &4&5 \\ \hline % or not
  1
      & 2 &3
& 4 4 4 4 4 4 4 4 & 5
\\
\hline % one row described in many lines

  \end{tabular}
\end{table}

\end{enumerate}
\end{document}
