%%%% BEGIN document formatting %%%%
\documentclass[11pt]{article}
\usepackage{fullpage}
\usepackage{setspace}
\onehalfspacing
\setlength{\topmargin}{-0.7in}
\setlength{\textwidth}{6.5in}
\setlength{\oddsidemargin}{0.0in}
\setlength{\textheight}{10.0in}
\setlength{\parindent}{0in}
\pagestyle{empty}
%%%% END document formatting %%%%

%%%% BEGIN math packages %%%%
\usepackage{amsmath}
\usepackage{amssymb}
\usepackage{mathtools}
%%%% END math packages %%%%

%%%% BEGIN pseudocode packages %%%%
\usepackage{algpseudocode,algorithm}
\usepackage{listings}
\lstset{
  columns=fixed,
  basicstyle=\ttfamily, % fixed-width font
}
%%%% END pseudocode packages %%%%

%%%% BEGIN other packages %%%%
\usepackage{hyperref}
%%%% END other packages %%%%

% This defines a new command which you can use to start a new problem item
% when under the enumerate environment, e.g.:
%
%   \problemitem{2}
%
%     Answer to problem 2...
%
\newcommand{\problemitem}[1]{
  \bigskip
  \item {\bf Solution to problem #1}
  \medskip
}


\begin{document}

\begin{flushright}
{\bf CSOR W4231.001---Summer 2019}
\end{flushright}
\begin{flushleft}
  Name: \\
  UNI: \\
  Session: {\bf R} (regular) / {\bf V} (CVN) \\ % keep the relevant choice
  Collaborators: \\ % list names and UNIs of all collaborators
\end{flushleft}

\bigskip
\centerline{\bf Homework 4}

\begin{enumerate}

\problemitem{1}

\problemitem{2}

\begin{enumerate}
  \item[(a)]

  \item[(b)]

\end{enumerate}

\end{enumerate}
\end{document}
